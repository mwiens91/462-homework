% Set up the document
\documentclass{article}

% Page size
\usepackage[
    letterpaper,]{geometry}

% Lines between paragraphs
\setlength{\parskip}{\baselineskip}
\setlength{\parindent}{0pt}

% Math
\usepackage{mathtools}
\usepackage{amssymb}
\usepackage{commath}

% Math notation macros
\def\*#1{\mathbf{#1}}
\newcommand{\dadvd}[2]{\dfrac{\text{D} #1}{\text{D} #2}} % advective derivative

\newcommand{\rhat}{\mathbf{\hat{r}}}
\newcommand{\thetahat}{\boldsymbol{\hat{\theta}}}
\newcommand{\xhat}{\mathbf{\hat{x}}}
\newcommand{\yhat}{\mathbf{\hat{y}}}
\newcommand{\zhat}{\mathbf{\hat{z}}}

% Links
\usepackage{hyperref}

% Page numbers at top right
\usepackage{fancyhdr}
\pagestyle{fancy}
\fancyhf{}
\fancyhead[R]{\thepage}
\renewcommand\headrulewidth{0pt}

% Graphics
\usepackage{float}
\usepackage{graphicx}
\graphicspath{ {./img/} }

\begin{document}

\textbf{MATH 462 Assignment 4} \\
\textbf{Matt Wiens \#301294492} \\
\textbf{2020-02-05}

\textbf{A) A Spinning Bucket Problem.}
Consider the steady flow velocity $\*u = (u, v, w) = (-\Omega y, \Omega x,
0)$. Verify that it is an incompressible flow. Explain clearly why it
corresponds to a uniformly rotating fluid (also known as a solid body
rotation). Then use the momentum equation (with gravity $\*F = - \rho g
\zhat$) to solve for the accompanying pressure field $p(x, y, z)$. As a
differential equation problem, explain why the condition $p(0, 0,
0) = p_{atm}$ is sufficient to uniquely determine the pressure.

Now discuss and resolve the apparent confusion suggested in the first
two paragraphs of Problem 1.2 (Acheson) involving the Bernoulli
streamline theorem (Section 1.3). (Extra: Why might an astronomer with
access to a lot of mercury find this result interesting?)

\textbf{Solution}

This flow is incompressible because
%
\begin{align*}
    \nabla \cdot \*u
        &= \dpd{}{x} \del{-\Omega y} + \dpd{}{y} \del{\Omega x} \\
        &= - y \dpd{\Omega}{x} + x \dpd{\Omega}{y} \\
        ????
\end{align*}

\newpage

\textbf{D) Streamlines \& Streamfunctions.}
The plotting assignment of Homework \#0 involved in the identification
of a streamfunction whose level curves correspond to level streamlines.
This problem explores another example of this idea.

In cylindrical coordinates, an axisymmetric flow has no dependence on
the $\theta$ variable. A Stokes streamfunction, $\Psi(r, z)$, provides a
steady, axisymmetric flow velocity having the form
%
\begin{equation*}
    \*u = U(r, z) \rhat + W(r, z) \zhat
\end{equation*}
%
through the derivative relations
%
\begin{equation*}
    U = - \frac{1}{r} \dpd{\Psi}{z}; \quad W = + \frac{1}{r} \dpd{\Psi}{r}
    .
\end{equation*}

\textbf{i)} For the specific case of the Stokes streamfunction
%
\begin{equation*}
    \Psi(r, z) = \frac{A}{2} r^2 + \frac{m}{4 \pi} \del{1 - \frac{z}{\sqrt{r^2 + z^2}}}
    ,
\end{equation*}
%
compute the velocity components, then verify they satisfy the
incompressibility condition.

\textbf{ii)} Show that this $\Psi(r, z)$ is constant along streamlines
by verifying that
%
\begin{equation*}
    \dadvd{\Psi}{t} = 0
    .
\end{equation*}
%
\textbf{iii)} Show that the flow direction is tangent to the streamline
contours everywhere, and note that the calculation is exactly the same
as in \textbf{ii)}. (Is this a coincidence?)

\textbf{iv)} Use the \texttt{contour} command in Matlab to produce a
graphic illustrating the flow by streamlines and flow direction arrows.
Annotate with parameter values, and all other relevant comments.

\textbf{v)} Find the exact location of the flow stagnation point (where
$\*U = \*0$). What is the value of the streamfunction there? Add this
special contour to your plot, and note that this contour can act as a
separating streamline of the flow (i.e., separating into ``inside'' and
``outside'' parts).

\textbf{vi)} Use the Bernoulli streamline theorem to give a formula for
the pressure deviation (assume that the pressure far upstream is a
constant, $p_{atm}$, and no body forces). Label regions of the
relatively high (H) and/or low (L) pressure regions on your flow graphic
(you may add these by hand, or plot the pressure field as an extra
Matlab plot). Verify that these regions are consistent with the observed
streamline curvatures.

\textbf{vii)} Basec on parts \textbf{i)}--\textbf{vi)}, is it clear that
this combination of velocity and pressure fields is a solution to the
incompressible Euler equations? Explain your reasoning.

\newpage

\textbf{Solution}

\textbf{i)} Taking $\Psi$ to be the Stokes streamfunction shown above,
we have that
%
\begin{align*}
    U &= - \frac{1}{r} \dpd{\Psi}{z}
        = \frac{m r}{4 \pi} \frac{1}{\del{r^2 + z^2}^{\frac{3}{2}}}
        , \\
    W &= + \frac{1}{r} \dpd{\Psi}{r}
        = A + \frac{m z}{4 \pi} \frac{1}{\del{r^2 + z^2}^{\frac{3}{2}}}
        .
\end{align*}
%
We can verify the incompressibility condition either by explicitly
computing the divergence or by noting that that mixed partial
derivatives commute. For the latter approach, we have that
%
\begin{equation*}
    \nabla \cdot \*u
        = \frac{1}{r} \dpd{\del{r U}}{r} + \dpd{W}{z}
        = \frac{1}{r} \dpd{}{r} \del{- \dpd{\Psi}{z}}
            + \dpd{}{z} \del{\frac{1}{r} \dpd{\Psi}{r}}
        = - \frac{1}{r} \dmd{\Psi}{2}{r}{}{z}{}
            + \frac{1}{r} \dmd{\Psi}{2}{z}{}{r}{}
        = 0
        .
\end{equation*}
%
\textbf{ii)} Now we will verify that $\Psi$ is constant along
streamlines. In the advective derivative
%
\begin{equation*}
    \dadvd{\Psi}{t} = \dpd{\Psi}{t} + (\*u \cdot \nabla) \Psi
    ,
\end{equation*}
%
the first term vanishes because $\Psi$ does not depend on $t$. The
second term also vanishes, since
%
\begin{equation*}
    (\*u \cdot \nabla) \Psi
        = \del{- \frac{1}{r} \dpd{\Psi}{z}, \frac{1}{r} \dpd{\Psi}{r}}
            \cdot \del{\dpd{\Psi}{r}, \dpd{\Psi}{z}}
        = - \frac{1}{r} \dpd{\Psi}{z} \dpd{\Psi}{r}
            +\frac{1}{r} \dpd{\Psi}{r} \dpd{\Psi}{z}
        = 0
        .
\end{equation*}
%
Therefore we have that
%
\begin{equation*}
    \dadvd{\Psi}{t} = 0
    ,
\end{equation*}
%
and we can conclude that $\Psi$ is constant along streamlines.

\textbf{iii)} To show that the flow direction is tangent to the
streamline contours everywhere, we need to show that $\*u$ is always
orthogonal to $\nabla \Psi$; that is, that $\*u \cdot \nabla \Psi = 0$
everywhere. This calculation is already done in step \textbf{ii)} above,
so we can conclude that the flow direction is indeed tangent to the
stream contours everywhere.

We saw in Homework 0 that the flow followed the streamline contours, so
this result should be expected.

((for parts $\geq$ iv) need to do plotting stuff))

\end{document}
