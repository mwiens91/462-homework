% Set up the document
\documentclass{article}

% Page size
\usepackage[
    letterpaper,]{geometry}

% Lines between paragraphs
\setlength{\parskip}{\baselineskip}
\setlength{\parindent}{0pt}

% Math
\usepackage{mathtools}
\usepackage{amssymb}
\usepackage{commath}

% Math notation macros
\def\*#1{\mathbf{#1}}
\newcommand{\B}{\mathcal{B}}
\newcommand{\rhat}{\mathbf{\hat{r}}}
\newcommand{\thetahat}{\mathbf{\hat{\theta}}}
\newcommand{\zhat}{\mathbf{\hat{z}}}

% Links
\usepackage{hyperref}

% Page numbers at top right
\usepackage{fancyhdr}
\pagestyle{fancy}
\fancyhf{}
\fancyhead[R]{\thepage}
\renewcommand\headrulewidth{0pt}

\begin{document}

\textbf{MATH 462 Homework 2} \\
\textbf{Matt Wiens \#301294492} \\
\textbf{2020-01-29}

\textbf{C) Lagrangian Derivation of Continuity.} Another common
derivation of the fluid equations relies on the integral identities of
multi-variable calculus---instead of the infinitesimal control volume
approach shown in lectures. Present a Lagrangian derivation of the
continuity equation based on the ideas and notation sketched out below.
Organize your write-up as a numbers list identifying each of the major
concepts/steps.

Consider any initial blob of fluid, $\B(0)$, and let $\B(t)$ Be the blob
of this original fluid as it moves with the flow. Note that we assume
the flow velocity is continuous so that we do not have to be concerned
with the fluid blob breaking into multiple bloblets.

Define the total fluid mass inside the blob by the volume integral over
the space occupied by the blob
%
\begin{equation*}
    M(t) = \iiint_{\B(t)} \rho(\*x, t) \dif x \dif y \dif z
\end{equation*}
%
and then explain how the principle of conservation of mass requires
%
\begin{equation*}
    \dod{M}{t} = 0
    .
\end{equation*}
%
Taking the derivative of the integral formula for $M(t)$ involves two
contributions: time-variations in the integrand $\rho(\*x, t)$, and
motion of the blob $\B(t)$. This differentiation is essentially a
three-dimensional version of Leibniz's rule. In particular, explain
explicitly how the second contribution can be expressed as a surface
integral. The basic idea is that over a time-interval $\Delta t$ the
change to the blob geometry can be accounted for by summing over the
displacements of small surface patches $\Delta S$. This surface integral
can be then converted to a volume integral using the divergence theorem.
Finally, invoke the zero-integrand property which states that if an
integral of the form
%
\begin{equation*}
    \iiint_{\B(t)} \cbr{\text{integrand}} \dif x \dif y \dif z = 0
\end{equation*}
%
for \textit{all} choices of $\B(t)$, then the integrand must be zero
everywhere.

\textbf{Solution}

five or six major steps. see diagrams

\newpage

\textbf{D) Polar Coordinates Again.} Present a control volume derivation
of the PDE representing Newton's law in fully three-dimensional cylindrical
coordinates $(r, \theta, z)$. Define the velocity as
%
\begin{equation*}
    \*U(r, \theta, z, t)
        = U(r, \theta, z, t) \rhat
          + V(r, \theta, z, t) \thetahat
          + W(r, \theta, z, t) \zhat
          ;
\end{equation*}
%
you are reminded that the $\rhat$ and $\thetahat$-directions are
functions of $\theta$. Present your results in acceleration form.

\textbf{Solution}

use diagrams etc

\end{document}
