% Set up the document
\documentclass{article}

% Page size
\usepackage[
    letterpaper,]{geometry}

% Lines between paragraphs
\setlength{\parskip}{\baselineskip}
\setlength{\parindent}{0pt}

% Math
\usepackage{mathtools}
\usepackage{amssymb}
\usepackage{commath}

% Short form for mathbf
\def\*#1{\mathbf{#1}}

% Links
\usepackage{hyperref}

% Page numbers at top right
\usepackage{fancyhdr}
\pagestyle{fancy}
\fancyhf{}
\fancyhead[R]{\thepage}
\renewcommand\headrulewidth{0pt}

\begin{document}

\textbf{MATH 462 Homework 0} \\
\textbf{Matt Wiens \#301294492} \\
\textbf{2020-01-15}

\textbf{A)} Consider a scalar function of two variables,
%
\begin{equation*}
    \psi(x, y) = y \del{1 - \frac{1}{r^2}} + \frac{B}{2} \ln(r^2)
    ,
\end{equation*}
%
where $r^2 = x^2 + y^2$ and $B$ is a positive constant. Define a vector
field $\*U(x, y) = \del{u(x, y), v(x, y)}$ where the scalar velocity
functions $u(x, y)$ and $v(x, y)$ are related to $\psi(x, y)$ by
%
\begin{equation*}
    u(x, y) = + \pd{\psi}{y}; \quad v(x, y) = - \pd{\psi}{x}.
\end{equation*}
%
Calculate these velocities and evaluate explicitly the divergence of the
velocity field. Carry out this calculation in (i) mixed Cartesian
variables $\{x, y, r\}$ where $r$ derivatives are done implicitly, and
in (ii) purely polar coordinates $\{r, \theta\}$. Explain why the end
result could have been anticipated, and that $\*U(x, y)$ can be
represented as a curl ($\nabla \times$).


Show that $\*U(x, y)$ can also be represented by a gradient, $\nabla
\phi(x, y)$, by constructing the simplest $\phi(x, y)$. Could this
result have been anticipated, or is there something special going on?
Your end result will seem to be multi-valued, but do you think this is a
serious issue?

\textbf{Solution}

hey. also: (Hint: if you have difficulty getting started, try setting B
= 0 first.)

\newpage

\textbf{B)} The posted Matlab script \textit{hw01.m} gives an example of
plotting level curves and vector fields. Modify the script to use the
$\psi(x, y)$ and $\*U(x, y)$ from problem A. The plot also
shows how to remove the points inside the unit circle---how does this
circle visually relate to the vector field? Note also that the plots
look qualitatively different over ranges of the parameter $B$.

Make two plots for values of $B$ that show two differing flow patterns.
Indicate by a (computed) red $\ast$ places where the velocity field is
$\*0$. Choose separate gridpoint sets for the level curves and the
vector fields to get visually useful and well-resolved plots.

\textbf{Solution}

hey

\end{document}
