% Set up the document
\documentclass{article}

% Page size
\usepackage[
    letterpaper,]{geometry}

% Lines between paragraphs
\setlength{\parskip}{\baselineskip}
\setlength{\parindent}{0pt}

% Math
\usepackage{mathtools}
\usepackage{amssymb}
\usepackage{commath}

% Math notation macros
\def\*#1{\mathbf{#1}}
\newcommand{\dadvd}[2]{\dfrac{\text{D} #1}{\text{D} #2}} % advective derivative

\newcommand{\fS}{\mathcal{S}} % fancy S
\newcommand{\tphi}{\tilde{\phi}}
\newcommand{\nhat}{\mathbf{\hat{n}}}
\newcommand{\rhat}{\mathbf{\hat{r}}}
\newcommand{\thetahat}{\boldsymbol{\hat{\theta}}}
\newcommand{\xhat}{\mathbf{\hat{x}}}
\newcommand{\yhat}{\mathbf{\hat{y}}}
\newcommand{\zhat}{\mathbf{\hat{z}}}
\newcommand{\omegavec}{\boldsymbol{\omega}}

% Links
\usepackage{hyperref}

% Page numbers at top right
\usepackage{fancyhdr}
\pagestyle{fancy}
\fancyhf{}
\fancyhead[R]{\thepage}
\renewcommand\headrulewidth{0pt}

% Graphics
\usepackage{float}
\usepackage{graphicx}
\graphicspath{ {./img/} }

\begin{document}

\textbf{MATH 462 Assignment 7} \\
\textbf{Matt Wiens \#301294492} \\
\textbf{2020-03-04}

\textbf{A) Flow past the wall.} (2 pages, 10pts) Consider a potential
flow as defined by the conformal map $w = \Phi(z)$, where $w = \phi + i
\psi$ and $\Phi(z) = \sqrt{z^2 + 1}$. The streamlines in the $z$-plane
are curves of constant imaginary $w$. Consider only the mapping of the
upper half $z$-plane to the upper half $w$-plane---this also uniquely
defines the branch of the square root.

\textbf{i)} Clearly state the branch choice of the square root (i.e.,
specify how to uniquely evaluate the square root). Then, determine the
pre-image of the $\Re(w)$-axis.

\textbf{ii)} Find the equation of the streamline curve in the $z$-plane
which is the pre-image of the coordinate line identified by $\Im(w) =
\psi$ in the $w$-plane. Express as an equation in the form $F(x, y;
\psi) = 0$.

\textbf{iii)} Then, find an equation of the curve in the $z$-plane which
is the pre-image of the coordinate line identified by $\Re(w) \phi$ in
the $w$-plane. Express as an equation of the form $G(x, y; \phi) = 0$.

\textbf{iv)} Show by an explicit calculation that these two pre-image
curves are orthogonal in the $z$-plane.

\textbf{Solution}

see figure in hw posting

\newpage

\textbf{B) Outflow from a source.} (3 pages, 10pts) Consider the
potential flow as defined by the complex potential
%
\begin{equation*}
    \Phi(z) = U z + \frac{Q}{2 \pi} \log z
\end{equation*}
%
for $z \neq 0$. The flow above is uniform flow past a source point
(plotted for $U = 1, Q= 2$). In this problem, you are asked to calculate
the volume flux (per unit height $z$) emanating from the origin in three
different ways:

\textbf{i)} by an integration involving the radial velocity on circles
$r = a$,

\textbf{ii)} by an integration of the complex potential on arbitrary
closed contours enclosing the origin exactly once, and

\textbf{iii)} by an explicit identification of the separating
streamlines (first, express the streamlines using the polar form $z = r
e^{i \theta}$). (You will have to address how the branch cut is
defined.)

Next, calculate the limiting gap $\Delta y$ between the separating
streamlines as $\Re(z) \to \infty$ in two different ways:

\textbf{iv)} by the streamline expression from \textbf{iii)}, and

\textbf{v)} by a deduction involving the volume flux.

\textbf{Solution}

see figure in hw posting

\end{document}
