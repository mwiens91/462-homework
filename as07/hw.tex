% Set up the document
\documentclass{article}

% Page size
\usepackage[
    letterpaper,]{geometry}

% Lines between paragraphs
\setlength{\parskip}{\baselineskip}
\setlength{\parindent}{0pt}

% Math
\usepackage{mathtools}
\usepackage{amssymb}
\usepackage{commath}

% Math notation macros
\newcommand{\R}{\mathbb{R}}

\def\*#1{\mathbf{#1}}
\newcommand{\dadvd}[2]{\dfrac{\text{D} #1}{\text{D} #2}} % advective derivative

\newcommand{\fS}{\mathcal{S}} % fancy S
\newcommand{\tphi}{\tilde{\phi}}
\newcommand{\nhat}{\mathbf{\hat{n}}}
\newcommand{\rhat}{\mathbf{\hat{r}}}
\newcommand{\thetahat}{\boldsymbol{\hat{\theta}}}
\newcommand{\xhat}{\mathbf{\hat{x}}}
\newcommand{\yhat}{\mathbf{\hat{y}}}
\newcommand{\zhat}{\mathbf{\hat{z}}}
\newcommand{\omegavec}{\boldsymbol{\omega}}

% Links
\usepackage{hyperref}

% Page numbers at top right
\usepackage{fancyhdr}
\pagestyle{fancy}
\fancyhf{}
\fancyhead[R]{\thepage}
\renewcommand\headrulewidth{0pt}

% Graphics
\usepackage{float}
\usepackage{graphicx}
\graphicspath{ {./img/} }

\begin{document}

\textbf{MATH 462 Assignment 7} \\
\textbf{Matt Wiens \#301294492} \\
\textbf{2020-03-04}

\textbf{A) Flow past the wall.} (2 pages, 10pts) Consider a potential
flow as defined by the conformal map $w = \Phi(z)$, where $w = \phi + i
\psi$ and $\Phi(z) = \sqrt{z^2 + 1}$. The streamlines in the $z$-plane
are curves of constant imaginary $w$. Consider only the mapping of the
upper half $z$-plane to the upper half $w$-plane---this also uniquely
defines the branch of the square root.

\textbf{i)} Clearly state the branch choice of the square root (i.e.,
specify how to uniquely evaluate the square root). Then, determine the
pre-image of the $\Re(w)$-axis.

\textbf{ii)} Find the equation of the streamline curve in the $z$-plane
which is the pre-image of the coordinate line identified by $\Im(w) =
\psi$ in the $w$-plane. Express as an equation in the form $F(x, y;
\psi) = 0$.

\textbf{iii)} Then, find an equation of the curve in the $z$-plane which
is the pre-image of the coordinate line identified by $\Re(w) = \phi$ in
the $w$-plane. Express as an equation of the form $G(x, y; \phi) = 0$.

\textbf{iv)} Show by an explicit calculation that these two pre-image
curves are orthogonal in the $z$-plane.

\newpage

\textbf{Solution}

\textbf{i)} Take the branch choice of the square root to be
%
\begin{equation*}
    \sqrt{z} = r^{\frac{1}{2}} e^{i \frac{\theta}{2}}, \quad \theta \in [0, 2 \pi)
    ,
\end{equation*}
%
then $\Phi$ takes $z$,
squares it---thus extending it to the full complex plane---shifts it to
the right by one, and then contracts it back to the upper half-plane.

Now, we'll find all $z$ such that $\Phi(z) \in \R$. Then, writing $z = x
+ i y$,  we have
%
\begin{equation*}
   \sqrt{x^2 - y^2 + 1 + i 2 x y} \in \R
   .
\end{equation*}
%
Then we require the imaginary part of the the argument of the square
root to vanish, and hence either $x = 0$ or $y = 0$. Also, the argument
of the square root needs to be non-negative. Thus we have
%
\begin{equation*}
    \text{pre-image of $\Re(w)$} = \Re(z)  \cup \cbr{z: 0 \leq \Im(z) \leq 1 \text{ and } \Re(z) = 0}
    ,
\end{equation*}
%
where we considered only the upper half $z$-plane.

\textbf{ii)} We can square both sides our equation for $w$
to get
%
\begin{equation}
    \phi^2 - \psi^2 + i 2 \phi \psi = x^2 - y^2 + 1 + i 2 x y
    \label{eq:a-sqrd}
    .
\end{equation}
%
Equating real and imaginary parts we can express
%
\begin{equation*}
    \phi = \frac{x y}{\psi}
\end{equation*}
%
and thus
%
\begin{equation*}
    \frac{x^2 y^2}{\psi^2} - \psi^2 = x^2 - y^2 + 1
    .
\end{equation*}
%
Hence we can write curves of constant $\psi$ as
%
\begin{equation*}
    F(x, y; \psi) = x^2 - y^2 + 1 + \psi^2 - \frac{x^2 y^2}{\psi^2} = 0
    .
\end{equation*}
%
\textbf{iii)} Now we'll perform a similar trick, instead this time for
curves of constant $\phi$. From~\eqref{eq:a-sqrd} we can express
%
\begin{equation*}
    \psi = \frac{x y}{\phi}
\end{equation*}
%
and thus
%
\begin{equation*}
    \phi^2 - \frac{x^2 y^2}{\phi^2} = x^2 - y^2 + 1
    .
\end{equation*}
%
Hence our curves of constant $\phi$ are given by
%
\begin{equation*}
    G(x, y; \phi) = x^2 - y^2 + 1 - \phi^2 + \frac{x^2 y^2}{\phi^2} = 0
    .
\end{equation*}
%
\textbf{iv)} First let's find where the curves $F$ and $G$ intersect.
Using Wolfram Alpha, $F(x, y; \psi) = G(x, y; \phi)$ implies
%
\begin{equation*}
    x = \pm \frac{\psi \phi}{y}, \quad y \neq 0, \quad \psi \phi \neq 0
    .
\end{equation*}
%
Computing $\nabla F$ and $\nabla G$, we have
%
\begin{align*}
    \nabla F &= \del{
        2 x \del{1 - \frac{y^2}{\psi^2}},
        - 2 y \del{1 + \frac{x^2}{\psi^2}}
    }, \\
    \nabla G &= \del{
        2 x \del{1 + \frac{y^2}{\phi^2}},
        - 2 y \del{1 - \frac{x^2}{\phi^2}},
    }
\end{align*}
%
and thus
%
\begin{align*}
    \nabla F \cdot \nabla G
        &= 4 x^2 \del{1 - \frac{y^2}{\psi^2}} \del{1 + \frac{y^2}{\phi^2}}
            + 4 y^2 \del{1 + \frac{x^2}{\psi^2}} \del{1 - \frac{x^2}{\phi^2}} \\
        &= 4 \frac{\psi^2 \phi^2 (x^2 + y^2) - x^4 y^2 - x^2 y^4}{\psi^2 \phi^2}
        .
\end{align*}
%
Using that $x y = \pm \psi \phi$ when the curves intersect, at the
intersection of the curves, the dot product of the gradients is
%
\begin{align*}
    \nabla F \cdot \nabla G
        &= 4 \frac{x^2 y^2 (x^2 + y^2) - x^4 y^2 - x^2 y^4}{x^2 y^2} \\
        &= 4 \frac{x^4 y^2 + x^2 y^4 - x^4 y^2 - x^2 y^4}{x^2 y^2} \\
        &= 0
        .
\end{align*}
%
This shows us that for all intersections of the pre-image curves, the
two curves are orthogonal.

\newpage

\textbf{B) Outflow from a source.} (3 pages, 10pts) Consider the
potential flow as defined by the complex potential
%
\begin{equation*}
    \Phi(z) = U z + \frac{Q}{2 \pi} \log z
\end{equation*}
%
for $z \neq 0$. The flow above is uniform flow past a source point
(plotted for $U = 1, Q= 2$). In this problem, you are asked to calculate
the volume flux (per unit height $z$) emanating from the origin in three
different ways:

\textbf{i)} by an integration involving the radial velocity on circles
$r = a$,

\textbf{ii)} by an integration of the complex potential on arbitrary
closed contours enclosing the origin exactly once, and

\textbf{iii)} by an explicit identification of the separating
streamlines (first, express the streamlines using the polar form $z = r
e^{i \theta}$). (You will have to address how the branch cut is
defined.)

Next, calculate the limiting gap $\Delta y$ between the separating
streamlines as $\Re(z) \to \infty$ in two different ways:

\textbf{iv)} by the streamline expression from \textbf{iii)}, and

\textbf{v)} by a deduction involving the volume flux.

\newpage

\textbf{Solution}

\textbf{i)} Expressing $z = r e^{i \theta}$, we can express $\Phi$ as
%
\begin{equation}
    \Phi(r, \theta) = U r \cos \theta + \frac{Q}{2 \pi} \log r
    + i \del{U r \sin \theta + \frac{Q}{2 \pi} \theta}
    \label{eq:b-i}
    .
\end{equation}
%
Here we'll take $\theta \in [0, 2 \pi)$. The radial velocity $v$ is
given by the $r$-component of $\nabla \phi = \nabla \Re(\Phi)$:
%
\begin{equation*}
    v = \dpd{\phi}{r} = U \cos \theta + \frac{Q}{2 \pi r}
    .
\end{equation*}
%
Hence the volume flux (per unit $z$-height) calculated on circles $r =
a$ is given by
%
\begin{equation*}
    \int_0^{2 \pi} v(a, \theta) (a \dif \theta)
        = \int_0^{2 \pi} \del{a U \cos \theta + \frac{Q}{2 \pi}} \dif \theta
        = Q
        .
\end{equation*}
%
\textbf{ii)} Now we'll calculate the same quantity using the Laurent
series of $\dod{\Phi^*}{z}$. Note that we identify $\dod{\Phi^*}{z}$ with
the velocity $\*u$. Now,
%
\begin{equation*}
    \dod{\Phi^*}{z}
        = \del{U + \frac{Q}{2 \pi z}}^*
\end{equation*}
%
Let $C$ be a contour enclosing the origin exactly once, oriented
\textit{clockwise} (this lets us drop the complex conjugate above).
Then, noting that $i \dif z$ represents the outward normal direction on
each segment of the contour $\dif z$, we have, using the Residue theorem
(modified for a clockwise contour), the velocity flux given by
%
\begin{equation*}
    \oint_C \dod{\Phi^*}{z} i \dif z
        = \oint_C \del{U + \frac{Q}{2 \pi z}} i \dif z
        = (- 2 \pi i) \frac{Q}{2 \pi} i
        = Q
        .
\end{equation*}
%
\textbf{iii)} Now we will calculate the velocity flux by identifying the
separating streamlines using~\eqref{eq:b-i}. As was discussed above,
we're taking the branch cut along $\theta = 0$. The streamfunction
identified in~\eqref{eq:b-i} is
%
\begin{equation*}
    \psi = U r \sin \theta + \frac{Q}{2 \pi} \theta
    .
\end{equation*}
%
To find the separating streamlines, we first need to find where the
velocity is $\*0$. This requires
%
\begin{equation*}
    \nabla \phi = \del{U \cos \theta + \frac{Q}{2 \pi r}, - U \sin \theta} = \*0
    .
\end{equation*}
%
which implies
%
\begin{equation*}
    \theta = \pi, \ r = \frac{Q}{2 \pi U}
    .
\end{equation*}
%
The value of the streamfunction at this point is clearly $\psi =
\frac{Q}{2}$. Contours of $\psi = \frac{Q}{2}$---the separating
streamlines---are thus given by
%
\begin{equation*}
    U r \sin \theta + \frac{Q}{2 \pi} \theta = \frac{Q}{2}
    .
\end{equation*}
%
Now, using our above formula for $\psi$, note that as $\theta \to 0$,
$\psi \to 0$; and as $\theta \to 2 \pi$, $\psi \to Q$. Hence, we have
that the volume flux per unit height above the real axis is
%
\begin{equation*}
    \Delta \psi = \frac{Q}{2} - 0 = \frac{Q}{2}
\end{equation*}
%
and below the real axis is
%
\begin{equation*}
    \Delta \psi = Q - \frac{Q}{2} = \frac{Q}{2}
\end{equation*}
%
In class we were told that the volume flux per unit height across an
area is $\Delta \psi$. Hence, summing the two $\Delta \psi$ terms we
have above (which cover the cross-sections of the area in the inner
region), we see that the volume flux per unit height is simply $Q$.

\textbf{iv)} Now we want to find the limiting gap $\Delta y$ between
separating streamlines. First, let's substitute $y = r \sin \theta$ into
the equation for the separating stream from above, and solve for y:
%
\begin{equation*}
    y = \frac{Q}{2 U} \del{1 - \frac{\theta}{\pi}}
\end{equation*}
%
Note that for the portion above the real axis, in the limit as $\Re(z)
\to \infty$, $\theta \to 0$; and below the real axis, in this same limit
we have that $\theta \to 2 \pi$. Hence we have that in the limit as
$\Re(z) \to \infty$,
%
\begin{equation*}
    \Delta y
        = y_{\text{above}} - y_{\text{below}}
        = \frac{Q}{2 U} \del{1 - \frac{0}{\pi}}
            - \frac{Q}{2 U} \del{1 - \frac{2 \pi}{\pi}}
        = \frac{Q}{U}
    .
\end{equation*}
%
\textbf{v)} First, notice that in our expression for the velocity
$\nabla \phi$ above, that as $\Re(z) \to \infty$ either $\theta \to 0$
or $\theta \to 2 \pi$, and in either case $\nabla \phi \to (U, 0)$.
Hence the volume flux in this limit must be $U \Delta y$. Combining this
with our above results, this implies that $\Delta y = \frac{Q}{U}$.

\end{document}
