% Set up the document
\documentclass{article}

% Page size
\usepackage[
    letterpaper,]{geometry}

% Lines between paragraphs
\setlength{\parskip}{\baselineskip}
\setlength{\parindent}{0pt}

% Math
\usepackage{mathtools}
\usepackage{amssymb}
\usepackage{commath}

% Math notation macros
\def\*#1{\mathbf{#1}}
\newcommand{\B}{\mathcal{B}}

% Links
\usepackage{hyperref}

% Page numbers at top right
\usepackage{fancyhdr}
\pagestyle{fancy}
\fancyhf{}
\fancyhead[R]{\thepage}
\renewcommand\headrulewidth{0pt}

\begin{document}

\textbf{MATH 462 Homework 1} \\
\textbf{Matt Wiens \#301294492} \\
\textbf{2020-01-22}

\textbf{A) Passive Drift.} Review the lecture notes for the passive
drift in 1D flow example
%
\begin{equation*}
    u(x) = \frac{1}{1 + a \sin x}
    .
\end{equation*}
%
This speed is not uniform in $x$, but we can still calculate a mean
flowspeed for the duck's motion since $u$ is spatially $2 \pi$-periodic.
Consider a duck starting at $X(0) = - \pi$. When it reaches $X(t_{per})
= + \pi$, the duck now sees the same downstream conditions as it did at
$t = 0$. To find the time when $X = + \pi$, use the first integral
function $t(X)$, so that $t_{per} = t(+\pi)$. The mean flowspeed over
the period is then calculated as
%
\begin{equation*}
    \text{mean flowspeed} = \frac{X_{per}}{t_{per}} = \frac{2 \pi}{t (+ \pi)}
\end{equation*}
%
and is (surprisingly) independent of the parameter $a$.

Now consider the time-dependent stream velocity
%
\begin{equation*}
    u(x, t) = \frac{1}{1 + a \sin (x - c t)}
    ,
\end{equation*}
%
where $c$ is the constant translation speed. Given the ODE problem for a
trajectory $X(t)$ starting at $X(0) = - \pi$. While this ODE is not of
the separable type, it is in the variable $Y(t) = X(t) - c t$. Show that
the separation method gives the form
%
\begin{equation*}
    (1 + a \sin Y) (1 - c F(Y)) \dod{Y}{t} = 1
    ,
\end{equation*}
%
where $F(Y)$ contains both $a$ and $c$ as parameters. Most importantly,
as unfriendly as the $F(Y)$-term looks, Wolfram Alpha knows how to
integrate it. As a check, the first integral result that you want
contains the combined real-valued parameter
%
\begin{equation*}
    A(a, c) = \sqrt{(1 - c)^2 - (ac)^2}
    .
\end{equation*}
%
Now comes the interpretive part. Obtain a formula for the time of
periodicity, $t_{per}(a, c)$, then get a formula for the mean flowspeed,
$X_{per}/t_{per}$. Wolfram Alpha gives a linear approximation in $c$
%
\begin{equation*}
    \frac{X_{per}}{t_{per}} \approx 1 - \frac{a^2 c}{2}
    .
\end{equation*}
%
Give an interpretation for why $c > 0$ causes a slowing of the drift.

\newpage

\textbf{Solution}

We can obtain $t(X)$ from
%
\begin{equation*}
    u(x) = \frac{1}{1 + a \sin x}
\end{equation*}
%
as follows:
%
\begin{align*}
    &\dod{X}{t} = \frac{1}{1 + a \sin X} \\
    &\implies \dif t = \del{1 + a \sin X} \dif X \\
    &\implies t(X) = \del{X - a \cos X} - \del{-\pi + a} \\
    &\implies t(X) = X - a \cos X + \pi - a
    .
\end{align*}
%
Then we have that $t_{per} = t(+\pi) =  2 \pi$, and hence the mean
flowspeed is given by $X_{per}/t_{per} = 2 \pi / t_{per} = 1$.

Now we will consider the time-dependent stream velocity (with $0 < c <
1$)
%
\begin{equation*}
    u(x, t) = \frac{1}{1 + a \sin (x - c t)}
    ,
\end{equation*}
%
which gives us the ODE
%
\begin{equation*}
    \dod{X}{t} = \frac{1}{1 + a \sin (X - c t)}
    .
\end{equation*}
%
If we let $Y(t) \coloneqq X(t) - c t$, then our equation becomes
%
\begin{equation*}
    \dod{Y}{t} + c = \frac{1}{1 + a \sin Y}
    ,
\end{equation*}
%
or, in ``separable form,''
%
\begin{align*}
    &\dod{Y}{t} = \frac{1}{1 + a \sin Y} - c \\
    &\implies \del{1 + a \sin Y} \dod{Y}{t} = 1 - c \del{1 + a \sin Y} \\
    &\implies \del{\frac{1}{1 - c \del{1 + a \sin Y}}} \del{1 + a \sin Y} \dod{Y}{t} = 1 \\
    &\implies \del{1 - c \sbr{\frac{1}{c} - \frac{1}{c \del{1 - c \del{1 + a \sin Y}}}}} \del{1 + a \sin Y} \dod{Y}{t} = 1 \\
    &\implies \del{1 - c \sbr{\frac{1 + a \sin Y}{a c \sin Y + c - 1}}} \del{1 + a \sin Y} \dod{Y}{t} = 1
    .
\end{align*}
%
If we write
%
\begin{equation*}
    F(x) \coloneqq \frac{1 + a \sin Y}{a c \sin Y + c - 1}
    ,
\end{equation*}
%
then our above ODE can be written
%
\begin{equation*}
    \del{1 - c F(Y)} \del{1 + a \sin Y} \dod{Y}{t} = 1
    .
\end{equation*}
%
Letting
%
\begin{equation*}
    A(a, c) \coloneqq \sqrt{(1 - c)^2 - (ac)^2}
    ,
\end{equation*}
%
we have that, according to Wolfram Alpha,
%
\begin{equation*}
    \int \del{1 - c F(Y)} \del{1 + a \sin Y} \dif {Y}
        = - \frac{1}{c A}
        \del{
            A Y + 2 \arctan
                \del{\frac{a c + (c - 1) \tan \del{\frac{Y}{2}}}{A}}
            } + \text{const}
    .
\end{equation*}
%
We can use this to solve our ODE. First let
%
\begin{equation*}
    G(x) \coloneqq
        - \frac{1}{c A}
        \del{
            A x + 2 \arctan
                \del{\frac{a c + (c - 1) \tan \del{\frac{x}{2}}}{A}}
            }
    ;
\end{equation*}
%
then we have
%
\begin{equation*}
    t(Y) = G(Y) - \lim_{x \to -\pi^+} G(x)
    .
\end{equation*}
%
By inspection, we can determine that that
%
\begin{align*}
    \lim_{x \to - \pi^+} G(x)
        &= - \frac{1}{c A} \del{- A \pi + 2 \frac{\pi}{2}} \\
        &= \frac{\pi}{c A} \del{A - 1}, \\ \\
    \lim_{x \to \pi^-} G(x)
        &= \frac{\pi}{c A} \del{1 - A}.
\end{align*}
%
Therefore,
%
\begin{equation*}
    t(Y) = G(Y) - \frac{\pi}{c A} \del{1 - A}
    ,
\end{equation*}
%
and also
%
\begin{equation*}
    t_{per}
        = \lim_{x \to \pi^-} t(x)
        = \lim_{x \to \pi^-} G(x) - \frac{\pi}{c A} \del{1 - A}
        = \frac{2 \pi}{c A} \del{1 - A}
    .
\end{equation*}
%
Therefore,
%
\begin{equation*}
    \frac{X_{per}}{t_{per}}
        = \frac{Y_{per} + c t_{per}}{t_{per}}
        = \frac{2 \pi}{\frac{2 \pi}{c A} \del{1 - A}} + c
        = \frac{c A}{\del{1 - A}} + c
    .
\end{equation*}
%
Using Wolfram Alpha, we can compute a linear approximation for this
result in $c$ to obtain
%
\begin{equation*}
    \frac{X_{per}}{t_{per}} \approx 1 - \frac{a^2 c}{2}
    .
\end{equation*}
%
Mathematically, the slowing of the drift is apparent when considering
that to reach any point $z$ in $Y$---that is, for some $t_z$ we have
$Y(t_z) = z$---to reach the corresponding point in $X$, we have $X(t_z)
- c t_z = z \implies X(t_z) = z + c t_z$. So in $X$ you can go further
in any time $t_z$ than in $Y$. Hence we have a ``slowing'' of the drift.


\newpage

\textbf{B) Polar Coordinates.} Present a control volume derivation of
the PDE representing conservation of mass for two-dimensional fluid flow
in three-dimensional cyclindrical coordinates $(r, \theta, z)$. In this
context, the two-dimensional flow is the special case with no vertical
flow ($w \equiv 0$) and no vertical variations ($\pd{}{z} \equiv 0$).
For convenience, define the velocities $U(r, \theta, t)$ and $V(r,
\theta, t)$ to be the flow components in the $\hat{r}$- and
$\hat{\theta}$-directions.

\textbf{Solution}

This one's done on paper.

\end{document}
