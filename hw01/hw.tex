% Set up the document
\documentclass{article}

% Page size
\usepackage[
    letterpaper,]{geometry}

% Lines between paragraphs
\setlength{\parskip}{\baselineskip}
\setlength{\parindent}{0pt}

% Math
\usepackage{mathtools}
\usepackage{amssymb}
\usepackage{commath}

% Math notation macros
\def\*#1{\mathbf{#1}}
\newcommand{\B}{\mathcal{B}}

% Links
\usepackage{hyperref}

% Page numbers at top right
\usepackage{fancyhdr}
\pagestyle{fancy}
\fancyhf{}
\fancyhead[R]{\thepage}
\renewcommand\headrulewidth{0pt}

% Graphics
% \usepackage{float}
% \usepackage{graphicx}
% \graphicspath{ {./img/} }

\begin{document}

\textbf{MATH 462 Homework 1} \\
\textbf{Matt Wiens \#301294492} \\
\textbf{2020-01-22}

\textbf{A) Passive Drift.} Review the lecture notes for the passive
drift in 1D flow example
%
\begin{equation*}
    u(x) = \frac{1}{a + a \sin x}
    .
\end{equation*}
%
This speed is not uniform in $x$, but we can still calculate a mean
flowspeed for the duck's motion since $u$ is spatially $2 \pi$-periodic.
Consider a duck starting at $X(0) = - \pi$. When it reaches $X(t_{per})
= + \pi$, the duck now sees the same downstream conditions as it did at
$t = 0$. To find the time when $X = + \pi$, use the first integral
function $t(X)$, so that $t_{per} = t(+\pi)$. The mean flowspeed over
the period is then calculated as
%
\begin{equation*}
    \text{mean flowspeed} = \frac{X_{per}}{t_{per}} = \frac{2 \pi}{t (+ \pi)}
\end{equation*}
%
and is (surprisingly) independent of the parameter $a$.

Now consider the time-dependent stream velocity
%
\begin{equation*}
    u(x, t) = \frac{1}{1 + a \sin (x - c t)}
    ,
\end{equation*}
%
where $c$ is the constant translation speed. Given the ODE problem for a
trajectory $X(t)$ starting at $X(0) = - \pi$. While this ODE is not of
the separable type, it is in the variable $Y(t) = X(t) - c t$. Show that
the separation method gives the form
%
\begin{equation*}
    (1 + a \sin Y) (1 - c F(Y)) \dod{Y}{t} = 1
    ,
\end{equation*}
%
where $F(Y)$ contains both $a$ and $c$ as parameters. Most importantly,
as unfriendly as the $F(Y)$-term looks. Wolfram Alpha knows hot to
integrate it. As a check, the first integral result that you want
contains the combined real-valued parameter
%
\begin{equation*}
    A(a, c) = \sqrt{(1 - c)^2 - (ac)^2}
    .
\end{equation*}
%
Now comes the interpretive part. Obtain a formula for the time of
periodicity, $t_{per}(a, c)$, then get a formula for the mean flowspeed,
$X_{per}/t_{per}$. Wolfram Alpha gives a linear approximation in $c$
%
\begin{equation*}
    \frac{X_{per}}{t_{per}} \approx 1 - \frac{a^2 c}{2}
    .
\end{equation*}
%
Give an interpretation for why $c > 0$ causes a slowing of the drift.

\textbf{Solution}

\newpage

\textbf{B) Polar Coordinates.} Present a control volume derivation of
the PDE representing conservation of mass for two-dimensional fluid flow
in three-dimensional cyclindrical coordinates $(r, \theta, z)$. In this
context, the two-dimensional flow is the special case with no vertical
flow ($w \equiv 0$) and no vertical variations ($\pd{}{z} \equiv 0$).
For convenience, define the velocities $U(r, \theta, t)$ and $V(r,
\theta, t)$ to be the flow components in the $\hat{r}$- and
$\hat{\theta}$-directions.

\textbf{Solution}

use diagrams. there's extra hints in the hw document

\newpage

\textbf{C) Lagrangian Derivation of Continuity.} Another common
derivation of the fluid equations relies on the integral identities of
multi-variable calculus---instead of the infinitesimal control volume
approach shown in lectures. Present a Lagrangian derivation of the
continuity equation based on the ideas and notation sketched out below.
Organize your write-up as a numbers list identifying each of the major
concepts/steps.

Consider any initial blob of fluid, $\B(0)$, and let $\B(t)$ Be the blob
of this original fluid as it moves with the flow. Note that we assume
the flow velocity is continuous so that we do not have to be concerned
with the fluid blob breaking into multiple bloblets.

Define the total fluid mass inside the blob by the volume integral over
the space occupied by the blob
%
\begin{equation*}
    M(t) = \iiint_{\B(t)} \rho(\*x, t) \dif x \dif y \dif z
\end{equation*}
%
and then explain how the principle of conservation of mass requires
%
\begin{equation*}
    \dod{M}{t} = 0
    .
\end{equation*}
%
Taking the derivative of the integral formula for $M(t)$ involves two
contributions: time-variations in the integrand $\rho(\*x, t)$, and
motion of the blob $\B(t)$. This differentiation is essentially a
three-dimensional version of Leibniz's rule. In particular, explain
explicitly how the second contribution can be expressed as a surface
integral. The basic idea is that over a time-interval $\Delta t$ the
change to the blob geometry can be accounted for by summing over the
displacements of small surface patches $\Delta S$. This surface integral
can be then converted to a volume integral using the divergence theorem.
Finally, invoke the zero-integrand property which states that if an
integral of the form
%
\begin{equation*}
    \iiint_{\B(t)} \cbr{\text{integrand}} \dif x \dif y \dif z = 0
\end{equation*}
%
for all choices of $\B(t)$, then the integrand must be zero everywhere.


\textbf{Solution}

five or six major steps. see diagrams

\end{document}
