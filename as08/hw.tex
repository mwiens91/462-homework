% Set up the document
\documentclass{article}

% Page size
\usepackage[
    letterpaper,]{geometry}

% Lines between paragraphs
\setlength{\parskip}{\baselineskip}
\setlength{\parindent}{0pt}

% Math
\usepackage{mathtools}
\usepackage{amssymb}
\usepackage{commath}

% Math notation macros
\newcommand{\R}{\mathbb{R}}
\newcommand{\Z}{\mathbb{Z}}

\def\*#1{\mathbf{#1}}
\def\ti#1{\tilde{#1}}
\newcommand{\dadvd}[2]{\dfrac{\text{D} #1}{\text{D} #2}} % advective derivative

\newcommand{\fS}{\mathcal{S}} % fancy S
\newcommand{\tphi}{\tilde{\phi}}
\newcommand{\nhat}{\mathbf{\hat{n}}}
\newcommand{\rhat}{\mathbf{\hat{r}}}
\newcommand{\thetahat}{\boldsymbol{\hat{\theta}}}
\newcommand{\xhat}{\mathbf{\hat{x}}}
\newcommand{\yhat}{\mathbf{\hat{y}}}
\newcommand{\zhat}{\mathbf{\hat{z}}}
\newcommand{\omegavec}{\boldsymbol{\omega}}

% Links
\usepackage{hyperref}

% Page numbers at top right
\usepackage{fancyhdr}
\pagestyle{fancy}
\fancyhf{}
\fancyhead[R]{\thepage}
\renewcommand\headrulewidth{0pt}

% Graphics
\usepackage{float}
\usepackage{graphicx}
\graphicspath{ {./img/} }

\begin{document}

\textbf{MATH 462 Assignment 8} \\
\textbf{Matt Wiens \#301294492} \\
\textbf{2020-03-18}

\textbf{A) Surface water wave in a layer.} (4 pages + plot)
Extend the 1D analysis from lecture to the geometry of a finite-depth
layer with an interface between two fluids characterized by different
densities, $\rho_1 < \rho_2$, where the lighter fluid is on top. For
the case of no waves, place the interface at $z = 0$ with the top layer
at $y = T$ and the bottom at $y = -B$. We will assume potential flow
(zero-divergence, constant density in each fluid, and irrotational) for
both fluids

i) Begin by explaining why we have two Bernoulli constants
$H_1 = p_0 / \rho_1$ and $H_2 = p_0 / \rho_2$, where $p_0$ is the
interface pressure in the absence of waves.

ii) For an interface displacement $y = \eta(x, t)$, you should be able
to state the complete nonlinear PDE problem for the surface water wave
problem.

iii) State the linearized PDE problem for the fixed geometry problem with
approximate interface conditions that are applied at $y = 0$.

iv) The solutions for sinusoidal potentials $\phi_1(x, y, t)$ and
$\phi_2(x, y, t)$ proceeds as discussed in class.

v) Obtain the dispersion relation $\omega(k)$ for waves having the form
$\sin(k x - \omega t)$.

vi) Discuss two features of your analysis: the effects to wavespeed on depth
of the lower fluid, and the frequency of the wave on density contrast
$\rho_1 / \rho_2$. Do your results hold in the limits $T, B \to \infty$,
and $\rho_1 / \rho_2 \to 0$?

\newpage

\textbf{Solution}

i) The reason we have two Bernoulli constants is because we can view the
problem from the top of the interface or from the bottom---essentially
two separate problems.

ii) Each ``Bernoulli II'' will give us an equation, while we also have
the equations coming from the advective derivatives. The system of
equations is as follows (which hold on $y = \eta(x, t)$:
%
\begin{equation*}
    \begin{dcases}
        \dpd{\phi_1}{t} + g \eta(x, t) + \frac{1}{2} |\*u_1|^2 = 0 \\
        \dpd{\phi_2}{t} + g \eta(x, t) + \frac{1}{2} |\*u_2|^2 = 0 \\
        v_1 - \eta_t - u_1 \eta_x = 0 \\
        v_2 - \eta_t - u_2 \eta_x = 0
    \end{dcases}
    .
\end{equation*}

iii) In linearizing the PDE, we throw out all quadratic (or greater)
perturbation terms, which leaves us with
%
\begin{equation*}
    \begin{dcases}
        \dpd{\phi_1}{t} + g \eta(x, t) = 0 \\
        \dpd{\phi_2}{t} + g \eta(x, t) = 0 \\
        v_1 - \eta_t = 0 \\
        v_2 - \eta_t = 0
    \end{dcases}
    .
\end{equation*}

This implies that $\phi_1 = \phi_2$ on $\eta(x, t)$, which is
consistent, but I'm definitely missing something. I don't think I know
how to interpret the problem correctly.

I will try and make sure I have someone reliable to talk to about the
homework next time.

\newpage

\textbf{B) Music of the sphere.} (3 pages)
This problem is based on \#3.13 in Acheson, and requires reading of section 3.6
of the chapter on waves.

i) Starting from the compressible Euler equations in coordinate-free
form (as in equations 3.55 and 3.56) with adiabatic thermodynamics, show
that the linearization about a quiet state $\rho = \rho_0$ and $\*u =
\*0$ gives a 3D wave equation for the density. Specialize this result
for a spherically symmetric wave.

ii) For a spherical shell of radius $L$, derive the eigenvalue relation
for the natural (temporal) frequencies, $\omega$, of the interior
standing waves
%
\begin{equation*}
    \tan \frac{\omega L}{c} = \frac{\omega L}{c}
    .
\end{equation*}
%
(The natural frequencies are associated with separation of variables
solutions having the form $\rho(r, t) = f(r) \cos(\omega t)$.) Explain
why the boundary conditions of bounded density at the origin and zero
velocity at $r = L$ are reasonable choices.

iii) Numerically calculate the first three eigenfunctions (as multiples of
$c / L$), and comment upon whether you think the \textit{music of this sphere}
would be a truly harmonious sound.

\newpage

\textbf{Solution}

i) The form of the Euler equations of motion we need to concern
ourselves with are
%
\begin{equation}
    \rho \del{\dpd{\*u}{t} + \*u \cdot \nabla \*u} = - \nabla p
    ,
    \label{eq:b-mom}
\end{equation}
%
and
%
\begin{equation}
    \dpd{\rho}{t} + \nabla \cdot \del{\rho \*u} = 0
    .
    \label{eq:b-cons}
\end{equation}
%
Additionally, assuming the gas is ``perfect'' gives us
%
\begin{equation}
    \dadvd{}{t} \del{p \rho^{-\gamma}} = 0
    ,
    \label{eq:b-adiab}
\end{equation}
%
where $\gamma$ is the thermodynamic constant discussed in lectures.

Assume that $\rho = \rho_0 (1 + \ti{\rho})$, $\*u = \ti{\*u}$, where
$\ti{\rho}$ and $\ti{\*u}$ are small. We will also assume pressure
variations are small; that is, $p = p_0 (1 + \ti{p})$. We can
use~\eqref{eq:b-adiab} to express the pressure perturbations $\ti{p}$ in
terms of $\ti{\rho}$. Since each fluid element initially has pressure
$p_0$ and density $\rho_0$,~\eqref{eq:b-adiab} implies
%
\begin{align*}
    &p_0 \rho_0^{- \gamma} = \del{p_0 (1 + \ti{p})} \del[1]{\rho_0 (1 + \ti{\rho})}^{- \gamma} \\
    &\iff (1 + \ti{p}) (1 + \ti{\rho})^{-\gamma} = 1
\end{align*}
%
Taking a binomial approximation we have $(1 + \ti{p}) (1 - \gamma
\ti{\rho}) = 1$, and, neglecting quadratic terms, this implies that
$\ti{p} = \gamma \ti{\rho}$.

Our goal is to use all of the equations to write
%
\begin{equation}
    \dpd[2]{\ti{\rho}}{t} = c^2 \nabla^2 \ti{\rho}
    ,
    \label{eq:b-wave}
\end{equation}
%
where $c$ is a to-be-determined constant.

Now, since $\*u \, (= \ti{\*u})$ is small, we can simplify (by
neglecting quadratic terms in our perturbations)~\eqref{eq:b-mom} to
%
\begin{equation}
    \rho_0 \dpd{\ti{\*u}}{t} = - \nabla p = - \gamma \nabla \rho
    ,
    \label{eq:b-mom-s}
\end{equation}
%
and~\eqref{eq:b-cons} to
%
\begin{equation}
    \dpd{\ti{\rho}}{t} + \rho_0 \nabla \cdot \ti{\*u} = 0
    .
    \label{eq:b-cons-s}
\end{equation}
%
Note that~\eqref{eq:b-cons-s} implies
%
\begin{equation}
    \nabla \cdot \ti{\*u} = - \frac{1}{\rho_0} \dpd{\ti{\rho}}{t}
    .
    \label{eq:b-cons-s2}
\end{equation}
%
Hence taking the divergence of~\eqref{eq:b-mom-s} we have
%
\begin{align*}
    &\rho_0 \dpd{}{t} \del{\nabla \cdot \ti{\*u}} - \gamma \nabla^2 \rho \\
    &\iff \dpd[2]{\ti{\rho}}{t} = \gamma \nabla^2 \rho
    ;
\end{align*}
%
note that we have assumed that $\pd{}{t}$ commutes with $\nabla \cdot$
above. Thus we have obtained~\eqref{eq:b-wave} where the wave speed $c =
\pm \sqrt{\gamma}$. If our wave is spherically symmetric then we only
need to worry about the $r$ derivatives and hence have
%
\begin{equation*}
    \dpd[2]{\ti{\rho}}{t} = c^2 \frac{1}{r^2} \dpd{}{r} \del{r^2 \dpd{\ti{\rho}}{r}}
    .
\end{equation*}

ii) Write $\rho(r, t) = f(r) \cos(\omega t)$ we can plug this into our
above equation and simplify to write
%
\begin{equation*}
    \omega^2 f + c^2 \frac{1}{r^2} \dod{}{r} \del{r^2 \dod{f}{r}} = 0
    ,
\end{equation*}
%
or, in the form,
%
\begin{equation*}
    \dod{\del{r^2 f^\prime(r)}}{r} = - k^2 r^2 f(r)
    .
\end{equation*}
%
where $k = \omega / c$. According to Wolfram Alpha, the solutions to
this equation are
%
\begin{equation*}
    f(r) = c_1 \frac{1}{r} \cos(k r) + c_2 \frac{1}{2 k r} \sin(k r)
    .
\end{equation*}
%
Hence we have that
%
\begin{equation*}
    \rho(r, t) = \cos(\omega t) \del{c_1 \frac{1}{r} \cos(k r) + c_2 \frac{1}{2 k r} \sin(k r)}
    .
\end{equation*}
%
Now we need to apply the boundary conditions. The boundary conditions we
are given are that $\rho = \rho_c$ at the origin and $\pd{\rho}{t} = 0$
at $r = L$. The first boundary condition means we must take $c_1 = 0$,
which leaves us with
%
\begin{equation*}
    \rho(r, t) = c_2 \frac{1}{2 k r} \cos(\omega t) \sin(k r)
    .
\end{equation*}
%
At $r = L$, the time derivative boundary is
%
\begin{equation*}
    - c_2 \frac{\omega}{2 k L} \sin(\omega t) \sin(k L) = 0
    .
\end{equation*}
%
This implies that $k = \omega / c = \frac{n \pi}{L}$ with $n \in \Z$.
This isn't consistent with the result I was meant to derive, so
I believe I've made an error above.

iii) The first three eigenfrequencies calculated numerically are
%
\begin{align*}
    \omega_0 &= 0, \\
    \omega_1 &\approx 4.4934 \frac{c}{L}, \\
    \omega_2 &\approx 7.7253 \frac{c}{L}, \\
    \omega_3 &\approx 10.9041 \frac{c}{L}
    .
\end{align*}
%
These sound great together! I played all frequencies together taking $c
/ L = 100 \text{Hz}$ and it made a nice chord.

\end{document}
