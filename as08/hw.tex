% Set up the document
\documentclass{article}

% Page size
\usepackage[
    letterpaper,]{geometry}

% Lines between paragraphs
\setlength{\parskip}{\baselineskip}
\setlength{\parindent}{0pt}

% Math
\usepackage{mathtools}
\usepackage{amssymb}
\usepackage{commath}

% Math notation macros
\newcommand{\R}{\mathbb{R}}

\def\*#1{\mathbf{#1}}
\newcommand{\dadvd}[2]{\dfrac{\text{D} #1}{\text{D} #2}} % advective derivative

\newcommand{\fS}{\mathcal{S}} % fancy S
\newcommand{\tphi}{\tilde{\phi}}
\newcommand{\nhat}{\mathbf{\hat{n}}}
\newcommand{\rhat}{\mathbf{\hat{r}}}
\newcommand{\thetahat}{\boldsymbol{\hat{\theta}}}
\newcommand{\xhat}{\mathbf{\hat{x}}}
\newcommand{\yhat}{\mathbf{\hat{y}}}
\newcommand{\zhat}{\mathbf{\hat{z}}}
\newcommand{\omegavec}{\boldsymbol{\omega}}

% Links
\usepackage{hyperref}

% Page numbers at top right
\usepackage{fancyhdr}
\pagestyle{fancy}
\fancyhf{}
\fancyhead[R]{\thepage}
\renewcommand\headrulewidth{0pt}

% Graphics
\usepackage{float}
\usepackage{graphicx}
\graphicspath{ {./img/} }

\begin{document}

\textbf{MATH 462 Assignment 8} \\
\textbf{Matt Wiens \#301294492} \\
\textbf{2020-03-18}

\textbf{A) Surface water wave in a layer.} (4 pages + plot)
Extend the 1D analysis from lecture to the geometry of a finite-depth
layer with an interface between two fluids characterized by different
densities, $\rho_1 < \rho_2$, where the lighter fluid is on top. For
the case of no waves, place the interface at $z = 0$ with the top layer
at $y = T$ and the bottom at $y = -B$. We will assume potential flow
(zero-divergence, constant density in each fluid, and irrotational) for
both fluids

i) Begin by explaining why we have two Bernoulli constants
$H_1 = p_0 / \rho_1$ and $H_2 = p_0 / \rho_2$, where $p_0$ is the
interface pressure in the absence of waves.

ii) For an interface displacement $y = \eta(x, t)$, you should be able
to state the complete nonlinear PDE problem for the surface water wave
problem.

iii) State the linearized PDE problem for the fixed geometry problem with
approximate interface conditions that are applied at $y = 0$.

iv) The solutions for sinusoidal potentials $\phi_1(x, y, t)$ and
$\phi_2(x, y, t)$ proceeds as discussed in class.

v) Obtain the dispersion relation $\omega(k)$ for waves having the form
$\sin(k x - \omega t)$.

vi) Discuss two features of your analysis: the effects to wavespeed on depth
of the lower fluid, and the frequency of the wave on density contrast
$\rho_1 / \rho_2$. Do your results hold in the limits $T, B \to \infty$,
and $\rho_1 / \rho_2 \to 0$?

\newpage

\textbf{Solution}

\newpage

\textbf{B) Music of the sphere.} (3 pages)
This problem is based on \#3.13 in Acheson, and requires reading of section 3.6
of the chapter on waves.

i) Starting from the compressible Euler equations in coordinate-free
form (as in equations 3.55 and 3.56) with adiabatic thermodynamics, show
that the linearization about a quiet state $\rho = \rho_0$ and $\*u =
\*0$ gives a 3D wave equation for the density. Specialize this result
for a spherically symmetric wave.

ii) For a spherical shell of radius $L$, derive the eigenvalue relation
for the natural (temporal) frequencies, $\omega$, of the interior
standing waves
%
\begin{equation*}
    \tan \frac{\omega L}{c} = \frac{\omega L}{c}
    .
\end{equation*}
%
(The natural frequencies are associated with separation of variables
solutions having the form $\rho(r, t) = f(r) \cos(\omega t)$.) Explain
why the boundary conditions of bounded density at the origin and zero
velocity at $r = L$ are reasonable choices.

iii) Numerically calculate the first three eigenfunctions (as multiples of
$c / L$), and comment upon whether you think the \textit{music of this sphere}
would be a truly harmonious sound.

\newpage

\textbf{Solution}

\end{document}
