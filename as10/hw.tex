% Set up the document
\documentclass{article}

% Page size
\usepackage[
    letterpaper,]{geometry}

% Lines between paragraphs
\setlength{\parskip}{\baselineskip}
\setlength{\parindent}{0pt}

% Math
\usepackage{mathtools}
\usepackage{amssymb}
\usepackage{commath}

% Math notation macros
\newcommand{\R}{\mathbb{R}}
\newcommand{\Z}{\mathbb{Z}}
\newcommand{\bs}{\text{BS}}
\newcommand{\uf}{\text{UF}}

\def\*#1{\mathbf{#1}}
\def\ti#1{\tilde{#1}}
\newcommand{\dadvd}[2]{\dfrac{\text{D} #1}{\text{D} #2}} % advective derivative

\newcommand{\fS}{\mathcal{S}} % fancy S
\newcommand{\tphi}{\tilde{\phi}}
\newcommand{\nhat}{\mathbf{\hat{n}}}
\newcommand{\rhat}{\mathbf{\hat{r}}}
\newcommand{\thetahat}{\boldsymbol{\hat{\theta}}}
\newcommand{\xhat}{\mathbf{\hat{x}}}
\newcommand{\yhat}{\mathbf{\hat{y}}}
\newcommand{\zhat}{\mathbf{\hat{z}}}
\newcommand{\omegavec}{\boldsymbol{\omega}}

% Links
\usepackage{hyperref}

% Page numbers at top right
\usepackage{fancyhdr}
\pagestyle{fancy}
\fancyhf{}
\fancyhead[R]{\thepage}
\renewcommand\headrulewidth{0pt}

% Graphics
\usepackage{float}
\usepackage{graphicx}
\graphicspath{ {./img/} }

\begin{document}

\textbf{MATH 462 Assignment 10} \\
\textbf{Matt Wiens \#301294492} \\
\textbf{2020-04-05}

\textbf{101) Rankine-Hugoniot Conditions.} (3 pages)
The global conservation property was demonstrated for the equation
$\rho_t + (\rho u)_x = 0$. It was then shown that a solution that was
piecewise regular with a moving discontinuity $x = S(t)$ resulted in a
jump condition---but some steps were skipped. Present the skipped step
by grouping the integration terms to isolate the Rankine-Hugoniot jump
condition. The jump condition from the conservation of momentum was
shown to be
%
\begin{equation*}
    S^\prime(t) \sbr{\rho u} = \sbr{\rho u^2 + p}
    ,
\end{equation*}
%
but was not derived. As in lecture, present the regular conservation
principle for momentum, and then derive the Rankine-Hugoniot jump
condition at a moving discontinuity.

Lastly, say we have a shock propagating to the right with $u_1 >
S^\prime > u_0$, so that the flow is slowing down across the shock.
Define upstream (to the right) conditions $(u_0, p_0, \rho_0)$, and
behind the shock conditions $(u_1, p_1, \rho_1)$. Solve for $p_1$ and
$\rho_1$ in terms of the other variables, and discuss the sign of the
jumps in these variables as the shock passes by.

\newpage

\textbf{Solution}

First we'll cover the basic conservation principles for the acoustic
equations, and then show what happens for the piston discussed in class
which goes from the piston point $x_p(t)$ to $L$ where we have for $t >
t_s$, a jump condition at $x = S(t)$ (the $t < t_s$ case with no jump
condition was show in lectures).

The conservation form of the acoustic equations, from lectures, is
%
\begin{equation*}
    \begin{dcases}
        \rho_t + \del{\rho u}_x = 0 \\[2pt]
        \del{\rho u }_t + \del{\rho u^2 + p}_x = 0
    \end{dcases}
    .
\end{equation*}
%
For $t > t_s$ in our piston we have a discontinuity at $x = S(t)$, and
hence we can write the global conservation of mass as
%
\begin{equation*}
    \int_{x_p}^S \del{\rho_t + \del{\rho u}_x} \dif x
        + \int_{S}^L \del{\rho_t + \del{\rho u}_x} \dif x
        = 0
        .
\end{equation*}
%
For the $t$ partial derivatives in the integrands we can interchange the
integral and derivative using the Leibniz rule; hence we have,
%
\begin{align*}
    \int_{x_p}^S \rho_t \dif x
        &= \dod{}{t} \del{\int_{x_p}^S \rho \dif x}
            + \rho(x_p, t) \dod{x_p}{t}
            - \rho_\bs(S, t) \dod{S}{t}
            ,
    \\
    \int_S^L \rho_t \dif x
        &= \dod{}{t} \del{\int_S^L \rho \dif x}
            + \rho_\uf(S, t) \dod{S}{t}
            .
\end{align*}
%
Note that $L$ does not depend on time, so the Leibniz term for $L$
vanished; also, for the densities on either side of the shock we have
used the ``BS'' and ``UF'' naming conventions from lecture. Using the
above results and writing
%
\begin{equation*}
    \dod{x_p}{t} = u(x_p, t)
    ,
\end{equation*}
%
we have
%
\begin{align*}
    0 &= \int_{x_p}^S \del{\rho_t + \del{\rho u}_x} \dif x
        + \int_{S}^L \del{\rho_t + \del{\rho u}_x} \dif x
    \\
    &=
    \int_{x_p}^S \rho_t \dif x
        + \int_{S}^L \rho_t \dif x
        + \int_{x_p}^S \del{\rho u}_x \dif x
        + \int_{S}^L \del{\rho u}_x \dif x
    \\
    &=
    \del{
        \dod{}{t} \del{\int_{x_p}^S \rho \dif x}
        + \rho(x_p, t) u(x_p, t)
        - \rho_\bs(S, t) \dod{S}{t}
    }
    + \del{
        \dod{}{t} \del{\int_S^L \rho \dif x}
        + \rho_\uf(S, t) \dod{S}{t}
    }
    \\
    &\qquad
    + \eval[2]{\del{\rho u}}_{x_p}^S
        + \eval[2]{\del{\rho u}}_{S}^L
\end{align*}
%
In the last line of the above equation note that the $\rho(L, t) u(L,
t)$ term vanishes because $u$ is $0$ in the undisturbed fluid, and we
also have cancellation in the $\rho(x_p, t) u(x_p, t)$ terms. Taking
these points into account and grouping terms, we thus have
%
\begin{equation*}
    \dod{}{t} \del{
        \int_{x_p}^S \rho \dif x
        + \int_S^L \rho \dif x
    }
    + \dod{S}{t} \del{\rho_\uf(S, t) - \rho_\bs(S, t)}
    + \del{\rho u}_\bs(S, t) - \del{\rho u}_\uf(S, t) = 0
    .
\end{equation*}
%
Because mass is conserved from one end of the piston to the end of its
container, the first term vanishes and we are left with, at $x = S$,
%
\begin{equation*}
    \dod{S}{t} \eval[2]{\rho}_\bs^\uf = \eval[2]{\rho u}_\bs^\uf
    .
\end{equation*}
%
I won't repeat this calculation for the momentum equation since the
steps are identical; however, I will comment on two (possibly
non-trivial) differences: (i) in the momentum case, we have cancellation
in the $\rho(x_p, t) u(x_p, t)$ terms; and (ii) we invoke that momentum
is conserved across the piston to the end of its container to justify
that
%
\begin{equation*}
    \dod{}{t} \del{
        \int_{x_p}^S \rho u \dif x
        + \int_S^L \rho u \dif x
    } = 0
    .
\end{equation*}
%
After carrying out the steps for the global conservation of momentum equation
%
%
\begin{equation*}
    \int_{x_p}^S \del{\del{\rho u}_t + \del{\rho u^2 + p}_x} \dif x
        + \int_S^L \del{\del{\rho u}_t + \del{\rho u^2 + p}_x} \dif x
        = 0
        .
\end{equation*}
%
(the same steps we performed for the global conservation of mass
equation) we are left with
%
\begin{equation*}
    \dod{S}{t} \eval[2]{\rho u }_\bs^\uf = \eval[2]{\del{\rho u^2 + p}}_\bs^\uf
    .
\end{equation*}
%
Now we're asked to consider what the values of $p_1$ and $\rho_1$ are
for the shock propagating to the right with $u_0 < S^\prime < u_1$.
Using our above Rankine-Hugoniot conditions, we have a system of two
equations in two unknowns:
%
\begin{equation*}
    \begin{dcases}
        S^\prime (\rho_1 - \rho_0) = \rho_1 u_1 - \rho_0 u_0 \\
        S^\prime (\rho_1 u_1 - \rho_0 u_0) = \rho_1 u_1^2 + p_1 - \rho_0 u_0^2 - p_0
    \end{dcases}
    .
\end{equation*}
%
Solving this system, we get
%
\begin{align*}
    p_1 &= \rho_0 (u_1 - u_0) (S^\prime - u_0) + p_0 \\
    \rho_1 &= \frac{\rho_0 (S^\prime - u_0)}{S^\prime - u_1}
    .
\end{align*}
%
Our conditions imply that the density is negative, so, as was discussed
in the online lecture today, there was obviously a mistake in the
question posted. In terms of the $p_1$ variable, it should be clear that
our conditions $u_0 < S^\prime < u_1$ guarantee $p_1 > p_0$.

\end{document}
