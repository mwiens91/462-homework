% Set up the document
\documentclass{article}

% Page size
\usepackage[
    letterpaper,]{geometry}

% Lines between paragraphs
\setlength{\parskip}{\baselineskip}
\setlength{\parindent}{0pt}

% Math
\usepackage{mathtools}
\usepackage{amssymb}
\usepackage{commath}

% Math notation macros
\newcommand{\R}{\mathbb{R}}
\newcommand{\Z}{\mathbb{Z}}
\newcommand{\bs}{\text{BS}}
\newcommand{\uf}{\text{UF}}

\def\*#1{\mathbf{#1}}
\newcommand{\dadvd}[2]{\dfrac{\text{D} #1}{\text{D} #2}} % advective derivative

\newcommand{\nhat}{\mathbf{\hat{n}}}
\newcommand{\rhat}{\mathbf{\hat{r}}}
\newcommand{\thetahat}{\boldsymbol{\hat{\theta}}}
\newcommand{\xhat}{\mathbf{\hat{x}}}
\newcommand{\yhat}{\mathbf{\hat{y}}}
\newcommand{\zhat}{\mathbf{\hat{z}}}
\newcommand{\omegavec}{\boldsymbol{\omega}}
\newcommand{\tauvec}{\boldsymbol{\tau}}

% Links
\usepackage{hyperref}

% Page numbers at top right
\usepackage{fancyhdr}
\pagestyle{fancy}
\fancyhf{}
\fancyhead[R]{\thepage}
\renewcommand\headrulewidth{0pt}

% Graphics
\usepackage{float}
\usepackage{graphicx}
\graphicspath{ {./img/} }

\begin{document}

\textbf{MATH 462 Assignment 11} \\
\textbf{Matt Wiens \#301294492} \\
\textbf{2020-04-11}

\textbf{103) Pipe flow.} (3 pages)
Review the notes from 3D Poiseuille pipe flow. The other exact solution
for this case is for flow in a pipe with an annulus cross-section, $S
\leq r \leq R$. The integration in $r$ now allows for keeping the singular term
at $r = 0$.

In addition to solving for the velocity $W(r)$, calculate the mass flux
and the global force balance. Be clear about all the signs in your shear
stress calculation---the cancellation you need is obvious, but the
understanding is in the proper tracing of the signs. Your presentation
of the stress terms will be paid special attention.

\newpage

\textbf{Solution}

Just as in the lecture notes, we will assume the flow is steady and
axisymmetric. Using the incompressible Navier-Stokes equations, we
deduced in lecture that pressure is a function of $z$ only and that the
$z$-velocity $W$ is a function of $r$ only, such that
%
\begin{equation*}
    p_z = - \frac{\Delta p}{L}
    ,
\end{equation*}
%
where $L$ is the length of the pipe, and that
%
\begin{equation}
    \frac{1}{r} \del{r W^\prime(r)}^\prime = - \frac{\Delta p}{\mu L}
    \label{eq:1-1}
    ,
\end{equation}
%
where the prime symbols denote $r$ derivatives. By integrating~\eqref{eq:1-1}
twice, and in doing so being clever to multiply by $r$ before the
first integration and divide by $r$ before the second integration, we
find that the general solution for $W$ is given by
%
\begin{equation*}
    W(r) = - \frac{\Delta p}{4 \mu L} r^2 + c_1 \log r + c_2
    ,
\end{equation*}
%
where $c_1$ and $c_2$ are constants that we need to determine.

To determine the constants $c_1$ and $c_2$ we will invoke ``no-slip''
boundary conditions at the walls of the pipe, so that $W(\pm S) = W(\pm
R) = 0$. These conditions give us that
%
\begin{equation*}
    - \frac{\Delta p}{4 \mu L} S^2 + c_1 \log S + c_2
    = - \frac{\Delta p}{4 \mu L} R^2 + c_1 \log R + c_2
    = 0
    .
\end{equation*}
%
This is a system of two equations in two unknowns which we can solve to
obtain (I used Maple here)
%
\begin{align*}
    c_1 &= \frac{\Delta p}{4 \mu L} \frac{R^2 - S^2}{\log\del{\frac{R}{S}}}, \\
    c_2 &= \frac{\Delta p}{4 \mu L} \frac{S^2 \log R - R^2 \log S}{\log\del{\frac{R}{S}}}
    .
\end{align*}
%
Hence, we can write the velocity $W$ as
%
\begin{align*}
    W(r) &= - \frac{\Delta p}{4 \mu L} r^2
        + \del{\frac{\Delta p}{4 \mu L} \frac{R^2 - S^2}{\log\del{\frac{R}{S}}}} \log r
        + \del{\frac{\Delta p}{4 \mu L} \frac{S^2 \log R - R^2 \log S}{\log\del{\frac{R}{S}}}} \\
         &= \frac{\Delta p}{4 \mu L} \del{S^2 - r^2 + \frac{(R^2 - S^2) \log\del{\frac{r}{S}}}{\log\del{\frac{R}{S}}}}
    .
\end{align*}
%
The simplification obtained in the second line of the above equation can
be obtained by adding zero in the form of
%
\begin{equation*}
    \frac{S^2 \log{S}}{\log\del{\frac{R}{S}}}
    - \frac{S^2 \log{S}}{\log\del{\frac{R}{S}}}
\end{equation*}
%
to the first line of the equation, and then the remaining steps in the
simplification are trivial.

Having determined an explicit formula for $W$, we can now calculate the
mass flux through the pipe with
%
\begin{align*}
    \text{mass flux}
        &= \rho_0 2 \pi \int_S^R W(r) r \dif r \\
        &= \frac{\pi \rho_0 \Delta p}{2 \mu L}
            \int_S^R
                \del{r S^2 - r^3 + r \frac{(R^2 - S^2) \log\del{\frac{r}{S}}}{\log\del{\frac{R}{S}}}}
            \dif r
        .
\end{align*}
%
There is quite a lot to evaluate in the integrand of the above equation,
so we'll do each term separately. For the first term we have
%
\begin{equation*}
    S^2  \int_S^R r \dif r = \frac{S^2 (R^2 - S^2)}{2}
    ;
\end{equation*}
%
for the second term,
%
\begin{equation*}
    \int_S^R r^3 \dif r = \frac{R^4 - S^4}{4}
    ;
\end{equation*}
%
and for the third term,
%
\begin{align*}
    \frac{R^2 - S^2}{\log\del{\frac{R}{S}}}
    \int_S^R r \log\del{\frac{r}{S}} \dif r
    &=
    \frac{R^2 - S^2}{\log\del{\frac{R}{S}}}
    \del{\frac{R^2}{4} \del{2 \log\del{\frac{R}{S}} - 1} + \frac{S^2}{4}}
    \\
    &= \frac{R^2 (R^2 - S^2)}{2} - \frac{(R^2 - S^2)^2}{\log\del{\frac{R}{S}}}
    .
\end{align*}
%
Hence, combining our results, we have
%
\begin{align*}
    \text{mass flux}
        &= \frac{\pi \rho_0 \Delta p}{2 \mu L}
            \del{
                \frac{S^2 (R^2 - S^2)}{2}
                - \frac{R^4 - S^4}{4}
                + \frac{R^2 (R^2 - S^2)}{2}
                - \frac{(R^2 - S^2)^2}{\log\del{\frac{R}{S}}}
            } \\
        &= \frac{\pi \rho_0 \Delta p}{8 \mu L}
            \del{
                R^4 - S^4
                - \frac{(R^2 - S^2)^2}{\log\del{\frac{R}{S}}}
            }
      .
\end{align*}
%
Moving forward to the shear stress, we need there to be a net zero force
in the $z$ direction at $r = S$ and $r = R$ and hence a shear stress to
balance the pressure force. Because the pressure force is in the
positive $z$ direction, the shear force should be in the negative $z$
direction. Calculating the shear stress, we have
%
\begin{align*}
    \tauvec_\rhat
        &= \mu W^\prime(r) \zhat \\
        &= \frac{\Delta p}{4 L} \dod{}{r} \del{S^2 - r^2 + \frac{(R^2 - S^2) \log\del{\frac{r}{S}}}{\log\del{\frac{R}{S}}}} \zhat \\
        &= \frac{\Delta p}{4 L} \del{- 2 r + \frac{R^2 - S^2}{r \log\del{\frac{R}{S}}}} \zhat
        .
\end{align*}

\end{document}
