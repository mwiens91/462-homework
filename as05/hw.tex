% Set up the document
\documentclass{article}

% Page size
\usepackage[
    letterpaper,]{geometry}

% Lines between paragraphs
\setlength{\parskip}{\baselineskip}
\setlength{\parindent}{0pt}

% Math
\usepackage{mathtools}
\usepackage{amssymb}
\usepackage{commath}

% Math notation macros
\def\*#1{\mathbf{#1}}
\newcommand{\dadvd}[2]{\dfrac{\text{D} #1}{\text{D} #2}} % advective derivative

\newcommand{\rhat}{\mathbf{\hat{r}}}
\newcommand{\thetahat}{\boldsymbol{\hat{\theta}}}
\newcommand{\xhat}{\mathbf{\hat{x}}}
\newcommand{\yhat}{\mathbf{\hat{y}}}
\newcommand{\zhat}{\mathbf{\hat{z}}}
\newcommand{\omegavec}{\boldsymbol{\omega}}

% Links
\usepackage{hyperref}

% Page numbers at top right
\usepackage{fancyhdr}
\pagestyle{fancy}
\fancyhf{}
\fancyhead[R]{\thepage}
\renewcommand\headrulewidth{0pt}

% Graphics
\usepackage{float}
\usepackage{graphicx}
\graphicspath{ {./img/} }

\begin{document}

\textbf{MATH 462 Assignment 5} \\
\textbf{Matt Wiens \#301294492} \\
\textbf{2020-02-12}

\textbf{A) A Patch of Vorticity.}
Consider an incompressible, but rotational, 2D fluid whose initial
condition is characterized by a circular patch of vorticity
%
\begin{equation*}
    \omega(\*x, 0) =
        \begin{dcases}
            \frac{1}{\pi a^2},& 0 \leq r < a \\
            0,& a < r < \infty
        \end{dcases}
        ,
\end{equation*}
%
where $\omega$ is the $\zhat$-component of vorticity $\omegavec$ and $r
= |\*x|$.

\textbf{i)} Solve for the initial streamfunction $\psi(\*x, 0)$, and
hence determine the initial flow velocity. Your task is simplified in
this geometry since the Poisson PDE for the streamfunction $\psi(\*x,
0)$ is really just an ODE in polar coordinates. Invoke the boundary
values that the flow is bounded at the origin, and decays to zero as $r
\to \infty$. It is also necessary to impose continuity on the
streamfunction and its associated flow at $r = a$. Choose the constant
part of the streamfunction to be zero at $r \to \infty$. Describe the
resulting flow pattern.

\textbf{ii)} Using the vorticity equation, deduce the time evolution of
this flow for $t > 0$. This result, in turn, makes it easy to determine
the pressure field. Invoke the conditions that the pressure approaches a
constant value $p^\infty$ as $r \to \infty$ and is continuous at $r =
a$. Explain why the pressure field is consistent with the flow pattern.

\textbf{iii)} Make a subplot or two showing the important flow
quantities (what should these be?) as a function of $r$.

\textbf{iv)} Finally, show that there is a limiting streamfunction as the patch
parameter $a \to 0$,
%
\begin{equation*}
    \psi_0(\*x, t) = \lim_{a \to 0} \psi(\*x, t)
    .
\end{equation*}
%
Acheson refers to this limit as the line vortex of 3D flow.

\textbf{v)} Evaluate two types of circulation integrals. The first are
circular loops centered at the origin with $r = R$. The second are
circular loops that do not enclose (or include) the origin. One of these
evaluations is a calculation; the other is a mathematical argument.

\newpage

\textbf{Solution}

\end{document}
